\documentclass{farlamp}

% latexmk -pvc -pdfxe -bibtex -interaction=nonstopmode -outdir=build draft-basis.tex

\addbibresource{references.bib}

\subject{Draft Basis}
\title{What I need for planning the Farlamp draft}
\author{Richard Möhn}
\date{\today}

\begin{document}
\maketitle
\RaggedRight

(For a project overview and a glossary, see the
\href{https://github.com/rmoehn/farlamp}{home page of the Farlamp
repository.})

The questions in the following are adapted from \textcite[p. 175]{CoR}. Some of
the section headings quote the chapter titles in that book.

%%%%%%%%%%%%%%%%%%%%%%%%%%%%%%%%%%%%%%%%%%%%%%%%%%%%%%%%%%%%
\section{Readers}

\paragraph{Who are my readers?}

\begin{itemize}
\item The members of LessWrong and the AI Alignment Forum. Perhaps the members
    of MIRIxDiscord.
\item Ideally conference or workshop attendants, or readers of a journal. I
    don't know if I can get my article into such circles, though.
\end{itemize}

\paragraph{What do they know?}

\begin{itemize}
\item Most know ML and CS better than I.
\item They might not know about IDA.
\end{itemize}

\paragraph{Why should they care about my problem?}

IDA is a major approach to AI alignment. How reliable it is, we don't know, and
therefore not whether and what precautions are needed. My research would provide
empirical evidence to help answer these questions.


%%%%%%%%%%%%%%%%%%%%%%%%%%%%%%%%%%%%%%%%%%%%%%%%%%%%%%%%%%%%
\section{Ethos}

\paragraph{What kind of ethos or character do I want to project?} From
\textcite[p. 119]{CoR}:

\begin{itemize}
\item ‘[support] claims with evidence that readers accept’
\item ‘[consider] issues from all sides’
\item ‘anticipate and address [readers'] questions and concerns’
\item ‘thoughtfully [consider] other points of view’
\item ‘acknowledge other views and explain [my] principles of reasoning in
    warrants’
\end{itemize}

→ ‘give readers good reason to work \emph{with} [me] in developing and testing
new ideas’


%%%%%%%%%%%%%%%%%%%%%%%%%%%%%%%%%%%%%%%%%%%%%%%%%%%%%%%%%%%%
\section{Question and answer}

\paragraph{Sketch my question and its answer in two or three sentences.}
Question: Can SupAmp or ReAmp stay reliable despite overseer failure?
The answer is to be determined.


%%%%%%%%%%%%%%%%%%%%%%%%%%%%%%%%%%%%%%%%%%%%%%%%%%%%%%%%%%%%
\section{Reasons and evidence}

Sketch the reasons and evidence supporting my claim. – TBD


%%%%%%%%%%%%%%%%%%%%%%%%%%%%%%%%%%%%%%%%%%%%%%%%%%%%%%%%%%%%
\section{Acknowledgements and responses}

\begin{itemize}
\item What questions, alternatives and objections are my readers likely to
    raise?
\item How do I respond to them?
\end{itemize}

TBD


%%%%%%%%%%%%%%%%%%%%%%%%%%%%%%%%%%%%%%%%%%%%%%%%%%%%%%%%%%%%
\section{Warrants}

\begin{itemize}
\item When may my readers not see the relevance of a reason to a claim?
\item Can I state the warrant that connects them?
\end{itemize}

TBD


\begin{FlushLeft}
% Couldn't find out within reasonable time how to make the formatting of titles
% uniform.
\printbibliography
\end{FlushLeft}
\end{document}
