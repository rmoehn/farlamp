\documentclass{farlamp}

% latexmk -pvc -pdfxe -bibtex -interaction=nonstopmode -outdir=build overfail2.tex
% ls * |entr latexmk -pdfxe -bibtex -interaction=nonstopmode -outdir=build overfail2.tex

\addbibresource{references.bib}

\subject{Analysis}
\title{Overseer failures in SupAmp and ReAmp}
\subtitle{How to determine their effect on overall failure rate}
\author{Richard Möhn}
\date{\today}

\begin{document}
\maketitle
\tableofcontents

\section{Introduction}

(For a general overview and a glossary, see the
\href{https://github.com/rmoehn/farlamp}{home page of the Farlamp
repository.})

Capability amplification can cause the failure rate of an ML system to blow up,
even when the amplified agent fails rarely. To fix this, reliability
amplification can be used. Alternatively, to keep the failure rate down and even
decrease it below the agent's, distillation by reinforcement learning (RL) might
prove sufficient. However, whether the result will be reliable enough, we don't
know.~\parencite{ChriRelAmp} But we do need to know in high-stakes situations
\parencite[see][]{ChriLearnCata}.

If RL alone is enough, we can avoid the computational and design overhead of
reliability amplification. Therefore, I'm planning to adapt to RL the iterated
distillation and amplification scheme from \textcite{CSASupAmp} (See also:
\href{https://github.com/rmoehn/farlamp/raw/master/supamp-reamp.pdf}{How to turn
SupAmp into ReAmp?}) Once adapted, I can investigate what overall failure rates
result from various overseer failure rates and other learning parameters.

In this preliminary analysis I lay out my current understanding of overseer
failure, related concepts and what I need to do. I will update it and answer
open questions on the way to implementation.


\section{Concepts}

Before I can write about my predictions, what experiments to run and how to
implement them, I will pin down some terms (bold) and concepts.


% Why am I writing all this? Because it hasn't been laid out clearly before.
% Paul introduces the problem alright, but he assumes a lot of knowledge about
% IDA and why he's talking about amplification at one point and adds in learning
% algorithms at another.

\subsection{General terms}

In the following, being more \Term{capable} means being able to answer harder
questions. The \Term{overseer} is a human or some extract of human judgment. The
\Term{assistant} is a machine learning (sub-)system. The assistant can answer
the overseer's questions. And the overseer can give data to the algorithm
that trains the assistant to become more capable.


\subsection{Iterated amplification and distillation}
\label{sec:iad}

The overseer alone can only provide data to train the assistant until their
capabilities are roughly equal. After this the training can be continued by
using \Term{Iterated distillation and amplification} (IDA). \Term{Distillation}
is the mechanism by which the assistant is trained. During \Term{amplification}
the assistant answers the overseer's questions to help it generate higher-level
training data.

IDA has been explained in other places \parencite{CotrIDA, ESSMLPIDA}. But in
order to write this analysis clearly I need the notion of amplification without
immediate distillation. For this I don't know of any brief explanation, so I've
written one.

Suppose you are an overseer $H$ who wants to train an assistant $X$
\parencite[symbols taken from][]{CSASupAmp} to give a solution to problem $p$.
If you want to train $X$ using supervised learning (SL), you ask yourself: ‘What
is the solution to $p$?’ You answer: ‘The solution is $s$.’ Then $(p, s)$ is the
training datum for $X$.

If you want to train $X$ using reinforcement learning, $X$ has to suggest a
solution $s$, then you ask yourself: ‘What reward should I give $X$ for solution
$s$ to problem $p$?’ You answer: ‘The reward should be $r$’. Then $(p, s, r)$ is
the training datum for $X$.

After some time $X$ can solve all the problems that you can solve. But you want
it to become able to solve harder problems, for which you alone can't answer the
question and therefore can't provide training data. What do you do? You break
the question down into sub-questions. These branch out into sub-sub-questions
and so forth, forming a tree. The further we go towards the leaves, the easier
the questions become. The leaf questions are so easy that we can answer them
without breaking them down further. Then the answers flow upwards and are
combined at each node until we obtain the root answer, the missing piece of the
training datum $(p, s)$ or $(p, s, r)$ for $X$. This is \Term{iterated
amplification} (IA), without distillation. (For examples of trees, see
\textcite{StuhFacCog}.)

However, this breaking down of questions is computationally expensive. What if
you had an assistant \X{-1} that is almost as capable as $X$? Then you could
break down only the top-level question into sub-questions, have them answered by
\X{-1}, and use these sub-answers to answer the root question. Again you've
obtained a training datum for $X$.

This would be the first iteration of iterated \Term{distillation} and
amplification: You train an agent \X{0} until it is roughly as capable as you.
The combination of you and the agent forms a more capable compound agent
\HX{0}. Then \HX{0} generates data to train the next agent \X{1}, which is
more capable than \X{0}. This is called ‘distilling’ \HX{0}. Now you can team
up with \X{1} to become \HX{1} and generate data to train \X{2}. And so on.

Now we add \Term{failure}. It's natural that you sometimes make a mistake and
return a wrong answer. Suppose you're doing IA. And a wrong sub-answer anywhere
in the tree leads to the root answer being wrong. And each non-leaf question
leads to at least two sub-questions. Then even if your failure probability is
very small, the probability of ending up with a wrong root answer approaches 1
exponentially quickly in the height of the tree.~\parencite{ChriRelAmp}

What happens with I\emph{D}A isn't so clear. It depends on how the $X^n$ are
trained. If the training algorithm causes them to faithfully copy all your
answers, the failure probability will compound as in IA. But if the training
algorithm makes the $X^n$ good at ‘recognizing concepts’, they will treat some
wrong inputs as outliers and the failure probability won't compound as quickly,
or will even go down as $n$ grows. Of course, the more mistakes you make, the
harder it becomes for a learner to tell apart right and wrong. The goal of the
Farlamp project is to find out how the training algorithm influences the failure
rate.


\subsection{Overseer determinism and non-determinism}

The overseer might behave and fail deterministically or non-deterministically.
Ie. it might have a fixed output for each input, and thus the same failures for
a specific set of inputs. Or it might return a random value for any input,
with a certain probability.

I can't rule out either way. While I tend to deduce non-determinism from the
overseer model in \textcite{ChriRelAmp}, this deduction is countered by the
section on sequential decision making. Further, there are realistic examples of
both ways: If the overseer is a learned agent, as $H'$ in \textcite{CSASupAmp},
it is mostly deterministic. Granted, the answers of $H'$ change over time,
because it is trained continuously on data from $H$. But I expect it to converge
on determinism. If the overseer is a human, as in Ought's factored evaluation
experiments \parencite{StuhDelCog}, it is non-deterministic. Although
determinism can again result if the human output is automated
\parencite[see][sec. ‘Caching’ f.]{StuhTaxCapAmp}.

% Why I tend to deduce non-determinism:
%
% If a sample from this distribution was drawn before a whole training episode,
% the whole training input might be adversarial. This appears less desirable
% than sampling a new overseer for each training step and therefore having
% adversity dispersed through the training. The latter corresponds to a
% non-deterministic overseer.
%
% Also sampling a "pure" overseer for a whole episode isn't possible. The result
% of the formalisation, the mixture, is realistic. The ingredients aren't.
% Because we wouldn't construct a completely adversarial policy. And we can't
% construct a completely benign one.
%
% Why sequential decision making counters: If the agents have to make a sequence
% of decisions, it sounds they're not resampled between decisions.
%
% Sometimes-adversarial deterministic policies are compatible with the
% "distribution A over policies", but not with the "view this as a simple
% mixture".


\subsection{Height of a question}

This property might become useful later in the process. The height of a
question is the hight of the tree the question would be decomposed into if
iterated amplification was used. The following definition doesn't depend on
trees and iterated amplification, but it uses the notion of primitive questions
from \textcite[app. C]{CSASupAmp}.

\begin{definition}[Height of a question] By recursion:
    \begin{enumerate}
    \item Primitive questions have height 0.
    \item A question has height $n$ if $n-1$ is the maximum height of a
        sub-question asked by $H$ to generate a response.
    \end{enumerate}
The response will be an answer in the case of SupAmp and a reward in the case of
ReAmp.
\end{definition}

\begin{example}
    The decomposition rule for \task{permutation powering} from \textcite[table
    3]{CSASupAmp} is:
    \begin{align}
        \sigma^{2k}(x) &= \sigma^k(\sigma^k(x)) \\
        \sigma^{2k+1}(x) &= \sigma(\sigma^k(\sigma^k(x)))
    \end{align}

    \begin{itemize}
        \item ‘What is $\sigma(4)$?’, has height 0, because it is a primitive
            question.
        \item ‘What is $\sigma^2(4)$?’, has height 1, because $H$ decomposes it
            into ‘What is $\sigma(4)$?’ and ‘What is $\sigma(\left<\text{answer
            to the first sub-question}\right>)$?’, which are primitive questions
            (height 0).
        \item ‘What is $\sigma^3(4)$?’, has height 1.
        \item ‘What is $\sigma^4(4)$?’, has height 2.
        \item ‘What is $\sigma^8(4)$?’, has height 3.
    \end{itemize}
\end{example}


\section{Predictions}
\label{sec:prediction}

Based on the previous sections I expect that if the overseer's failure
probability is above a certain threshold $\hat{p}$ (call it the failure
tolerance), IDA increases the overall failure probability. If the overseer's
failure probability is below the threshold, IDA decreases the overall failure
probability. \textcite{ChriRelAmp} might have expressed this before, but I don't
quite understand the relevant sections.

I expect that the threshold depends on a few parameters, such as whether $H$ is
deterministic and how $X$ is trained. I tried to predict in which direction each
parameter pushes the threshold. But so far my ideas have led me into the fog,
because I know too little about the details of SL and RL.

For example, \textcite{ChriRelAmp} reasons: ‘if the overseer fails with
probability $1 \%$, then this only changes the reward function by 0.01, and an
RL agent should still avoid highly undesirable actions’. Can't the same be said
for SL? If only $1 \%$ of the training data for $X$ are wrong, won't it treat
them as outliers, provided enough regularization?

Note that \textcite{ChriRelAmp} doesn't state anything about SL. He only
predicts that distillation by imitation learning (IL) will lower reliability. I
imagine IL similar to a lookup table, so it makes sense that it will replicate
$H$'s failures (see sec. \ref{sec:iad}). But a lookup table would not be able
hold all of the possible input-output pairs of the tasks in \textcite{CSASupAmp}
anyway. Instead, the learning algorithm needs to ‘understand’ what the tasks are
about. This in turn makes it more likely to discard faulty data from the
overseer.

I do think the failure tolerance $\hat{p}$ will be greater with a
non-deterministic overseer than with a deterministic one. Because in the former
case a randomly wrong training datum will be overriden by a correct one, but in
the latter the wrong datum would be repeated – hammered in, so to say. One might
argue that such repetitions are rare if the domain is large. But the domains
start out fairly small in the training schedule of \textcite{CSASupAmp}. And the
start is where the foundation for the rest of the training is laid. If this
foundation is faulty, correct results cannot be built.


\section{Experiments}

I want to measure the overall failure rate of SupAmp and ReAmp, given various
overseer failure rates and maximum question heights, a non-deterministic and a
deterministic overseer, and perhaps different degrees of regularization. I will
decide the details later.

In the non-deterministic case overseer failure can be modelled by injecting a
random solution (SupAmp) or random evaluation (ReAmp) into the output of $H'$
with a certain probability. One might think that not $H'$, but $H$ should fail,
because the former is the actual overseer and the latter its imitation. But this
would just degrade the learning performance of $H'$ and not deliver randomly
wrong training inputs to $X$ as required.

The deterministic case is similar, except that a small fixed set of inputs to
$H'$ must be mapped to fixed random outputs. This can be done with a lookup
table. Or by hashing the input to provide a fail/no fail value and in the ‘fail’
case obtaining a fixed random output by hashing again.


\begin{FlushLeft}
% Couldn't find out within reasonable time how to make the formatting of titles
% uniform.
\printbibliography
\end{FlushLeft}
\end{document}
