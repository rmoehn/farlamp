\documentclass{farlamp}

% latexmk -pvc -pdfxe -bibtex -interaction=nonstopmode -outdir=build literature.tex

\addbibresource{references.bib}
\DeclareBibliographyCategory{annotated}

\author{Richard Möhn}
\date{\today}

\title{Literature overview}
\addtitledatatopdf

\begin{document}
\maketitle
\tableofcontents

\section{To search}

\begin{itemize}
    \item What has Paul written about RL-based IDA?
    \item How are rewards determined in RL?
    \item Since I mention IL, read something about IL?
    \item Find something about how ML algorithms deal with faulty data/outliers.
    \item \done\ See todos in SupAmp-ReAmp and Overfail2
    \item \done\ \href{https://scholar.google.com/scholar?hl=en&as_sdt=0%2C5&q=supervising+strong+learners+by+amplifying+weak+experts&btnG=}{Search
        backward from \textcite{CSASupAmp} on Google Scholar}
    \item Something about surrogate modelling? It appears to be related with
        distillation.
\end{itemize}


\section{Potential sources}

\begin{itemize}
    \item Possibly relevant works citing \textcite{CSASupAmp}, according to
        Google Scholar:
        \begin{itemize}
            \item \href{https://scholar.google.com/scholar?hl=en&as_sdt=0%2C5&q=supervising+strong+learners+by+amplifying+weak+experts&btnG=}{Backward search on Google Scholar}
            \item \href{https://arxiv.org/abs/1906.08663}{Modeling AGI Safety Frameworks
                with Causal Influence Diagrams}
            \item \href{https://arxiv.org/abs/1906.10189}{Evolutionary
                Computation and AI Safety: Research Problems Impeding Routine
                and Safe Real-world Application of Evolution}
            \item \href{https://www.mdpi.com/2504-2289/3/2/21}{Multiparty
                Dynamics and Failure Modes for Machine Learning and Artificial
                Intelligence}
            \item \href{https://arxiv.org/abs/1906.01820}{Risks from Learned Optimization in Advanced Machine Learning Systems}
        \end{itemize}
\end{itemize}


\section{To skim and decide}

\begin{itemize}
\item Resources from \textcite{CSASupAmp} that I've marked with a blue cross.
\item
    \href{https://ai-alignment.com/semi-supervised-reinforcement-learning-cf7d5375197f}{Semi-supervised reinforcement learning}
\item \href{https://arxiv.org/abs/1811.07871}{Scalable agent alignment via
    reward modeling: a research direction}
\end{itemize}


\section{To (re-)read}

\begin{itemize}
    \item \cite{ChriREngP}
    \item \cite{ChriThoRewE}
\end{itemize}


\section{Annotated bibliography}

\begin{displayquote}[{\cite[p. 102 f.]{CoR}}]
    Often the assembling of an annotated bibliography is a distinct stage in a
    research process […]. Each annotation is an opportunity to evaluate the
    credibility of a source, summarize its argument, and explain its relevance
    to your project.

    […] If you can't summarize your sources or explain their relevance, you are
likely not ready to write your paper.
\end{displayquote}

\annotitem{ChriRelAmp}
TODO: Copy summary from notes and clean up. Add relevance.

\annotitem{CSASupAmp}
TODO: Copy summary from notes and clean up. Add relevance.


\begin{FlushLeft}
    \printbibliography[notcategory=annotated]
\end{FlushLeft}
\end{document}


% NEXT:
% - List more sources. – From from Google Scholar, actually from all the points
%   above. They're not fully expanded yet.
% - For each source, decide when it makes sense to read it. Much of it only
%   makes sense after I've learned more about ML.
